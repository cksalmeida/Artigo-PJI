O presente projeto tem como objetivo geral o desenvolvimento de um sistema inteligente de monitoramento de sono voltado à prevenção de Acidente Vascular Cerebral (AVC) isquêmico em idosos com risco cardiovascular elevado. O sistema visa realizar a coleta, análise e disponibilização contínua de sinais vitais, com ênfase na frequência e ritmo cardíaco durante o período de repouso, a fim de possibilitar o acompanhamento médico preciso e preventivo.\\

Como objetivos específicos, destacam-se:\\

\begin{enumerate}[itemsep=1em]
    \item Desenvolver um dispositivo de monitoramento contínuo capaz de captar sinais vitais de forma não invasiva e precisa durante o sono;
    \item Registrar, armazenar e organizar os dados coletados em um ambiente seguro, possibilitando o acompanhamento longitudinal da saúde do paciente;
    \item Processar e analisar os dados para identificar variações significativas ou padrões que possam indicar risco de arritmias cardíacas associadas ao AVC isquêmico;
    \item Disponibilizar os resultados de forma acessível e interpretável ao profissional de saúde, fornecendo suporte à tomada de decisões clínicas relacionadas à prevenção e intervenção precoce;
    \item Contribuir para a redução da incidência e mortalidade associadas ao AVC, promovendo um modelo de cuidado mais contínuo, tecnológico e centrado na prevenção.\\
\end{enumerate}

Dessa forma, o projeto busca integrar tecnologia, ciência de dados e prática médica, oferecendo ao profissional de saúde uma ferramenta capaz de otimizar o diagnóstico preventivo e aperfeiçoar o processo decisório clínico. A interpretação automatizada e a disponibilização visual dos dados ampliam a capacidade do médico em identificar riscos de forma antecipada, impactando diretamente na qualidade de vida e na sobrevida dos pacientes idosos.