Nas últimas décadas, o monitoramento remoto de sinais vitais tem se consolidado como uma das vertentes mais promissoras da tecnologia aplicada à saúde, especialmente no contexto da medicina preventiva e do envelhecimento populacional. O avanço dos sensores ópticos, da Internet das Coisas (IoT) e da Inteligência Artificial permitiu o desenvolvimento de sistemas capazes de captar e interpretar dados fisiológicos em tempo real, possibilitando a identificação precoce de anomalias cardíacas associadas a eventos cerebrovasculares, como o Acidente Vascular Cerebral (AVC) isquêmico.

Entre as abordagens mais promissoras para o monitoramento contínuo da frequência e do ritmo cardíaco destaca-se a fotopletismografia (PPG), uma técnica óptica não invasiva que mensura variações no volume sanguíneo por meio da luz refletida ou transmitida pelos tecidos biológicos. Essa metodologia tem se mostrado altamente eficaz na detecção precoce de arritmias, em especial da fibrilação atrial (FA). O estudo conduzido por \textcite{Lubitz2022}, publicado na revista Circulation, avaliou a aplicação de algoritmos de detecção de FA baseados em PPG, demonstrando elevada sensibilidade e especificidade na identificação de eventos arrítmicos por meio de dispositivos vestíveis. Os resultados evidenciaram o potencial clínico dessa tecnologia para o rastreamento contínuo e não supervisionado de distúrbios cardíacos, reforçando sua relevância em estratégias de prevenção secundária de AVC e na promoção de uma vigilância cardíaca remota e acessível.

Apesar da precisão obtida com a fotopletismografia, a interpretação manual dos sinais ainda representa um desafio clínico devido ao grande volume de dados gerados. Nesse cenário, o avanço de técnicas de aprendizado profundo baseadas em Inteligência Artificial surge como resposta ao desafio da análise automatizada e em tempo real desses sinais. Pesquisas recentes, como a de \textcite{Charlton2023}, demonstram que o emprego de redes neurais profundas na interpretação de sinais fotopletismográficos possibilita a detecção em tempo real de anomalias cardíacas e respiratórias com elevada acurácia, mesmo em ambientes domésticos. Essa capacidade de processamento dinâmico permite identificar padrões sutis e preditivos, viabilizando um rastreamento proativo de eventos cardiovasculares.

Em paralelo, o estudo conduzido por \textcite{DeVries2023} reforça a confiabilidade dos dispositivos vestíveis na aquisição contínua de dados de sono e frequência cardíaca em contextos não controlados, validando sua aplicabilidade clínica. A sinergia entre inteligência artificial e biossensoriamento representa, portanto, um avanço significativo rumo à assistência médica preventiva e personalizada, em que o monitoramento remoto fornece subsídios objetivos para tomadas de decisão clínicas baseadas em dados.

Contudo, apesar dos avanços, observa-se que a maioria das soluções existentes tem foco na automonitorização pelo próprio paciente, sem uma integração direta com o profissional de saúde responsável pelo acompanhamento clínico. Nesse contexto, o projeto proposto neste trabalho diferencia-se ao centralizar o médico como principal usuário final do sistema, oferecendo uma plataforma de apoio à interpretação diagnóstica e à tomada de decisão terapêutica, transformando dados de sinais vitais em informações clínicas de alto valor preditivo.

A proposta, portanto, alinha-se às tendências mais recentes de telemonitoramento e medicina personalizada, mas introduz um enfoque inovador ao unir o monitoramento contínuo do sono, a análise inteligente de anomalias cardíacas e o suporte direto à decisão médica, com o objetivo de aprimorar a prevenção de AVC em idosos e reduzir a mortalidade associada a eventos cerebrovasculares.