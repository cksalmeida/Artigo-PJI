A presente pesquisa reafirma o papel fundamental da tecnologia como instrumento catalisador na prevenção e monitoramento de doenças cardiovasculares, especialmente no contexto do envelhecimento populacional e do risco aumentado de Acidente Vascular Cerebral (AVC). O projeto de Monitoramento Inteligente do Sono identifica uma lacuna crítica na interseção entre análise de sinais vitais e apoio à decisão clínica, propondo uma solução concreta, sustentada em fundamentos científicos sólidos e em experimentação laboratorial voltada à validação técnica do sistema.

O protótipo desenvolvido, composto por um sensor óptico de baixo custo integrado a uma infraestrutura modular baseada em API Nest.js, banco de dados MongoDB e interface em Next.js, demonstra a viabilidade de um sistema capaz de coletar e interpretar parâmetros fisiológicos com confiabilidade. Mais do que um dispositivo de monitoramento, o sistema se configura como uma ferramenta de suporte médico, oferecendo dados clínicos objetivos que podem subsidiar a tomada de decisões preventivas e terapêuticas voltadas à redução da incidência de AVC em idosos.

Importa destacar que o projeto não se propõe a substituir o acompanhamento médico tradicional, mas a complementá-lo, promovendo uma integração inteligente entre tecnologia e prática clínica. Essa convergência potencializa o diagnóstico precoce, amplia a eficiência da vigilância de pacientes de risco e reforça o papel do profissional de saúde como agente central na interpretação dos dados. Assim, o sistema contribui para a consolidação de um modelo de medicina mais preditiva, personalizada e orientada por dados.

Entre as perspectivas futuras, destaca-se a transição do protótipo com conexão física via fio para uma versão totalmente sem fio, visando maior conforto e aplicabilidade no ambiente doméstico. Também se prevê a integração de algoritmos de Inteligência Artificial capazes de identificar padrões fisiológicos indicativos de risco de AVC, bem como a ampliação da base de dados para treinamento e validação desses modelos. Em paralelo, pretende-se aprimorar a interface de visualização e os relatórios médicos, garantindo maior precisão e usabilidade no acompanhamento clínico remoto.

Dessa forma, o Monitoramento Inteligente do Sono transcende sua dimensão técnica e assume um papel estratégico na promoção da saúde, longevidade e qualidade de vida da população idosa, ao alinhar inovação tecnológica, rigor científico e responsabilidade social. O projeto consolida-se, assim, como uma contribuição significativa para o avanço da medicina preventiva e para o fortalecimento das práticas de cuidado baseadas em evidências e conectividade digital.

Adicionalmente, a iniciativa está alinhada aos Objetivos de Desenvolvimento Sustentável da ONU, em especial a ODS 3 – Saúde e Bem-Estar, ao promover soluções que visam reduzir a mortalidade por doenças cardiovasculares, garantir acesso a cuidados preventivos e apoiar a qualidade de vida de populações vulneráveis. Nesse sentido, o projeto demonstra como a convergência entre tecnologia, ciência e políticas de saúde pode contribuir para metas globais de promoção da saúde, equidade e inclusão.